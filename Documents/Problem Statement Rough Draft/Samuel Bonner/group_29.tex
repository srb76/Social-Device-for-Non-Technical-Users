\documentclass[letterpaper,10pt,fleqn,draftclsnofoot,onecolumn]{article}

\def\name{Problem Statement}

\parindent = 0.0 in
\parskip = 0.2 in

\title{Problem Statement - CS461}
\author{Samuel Bonner - Fall 2017}

\begin{document}
\maketitle
\hrule

\section*{Abstract}

NOTE: Due to the rescheduling of a sponsor meeting, group 29 has not yet met with their sponsor, Arian Kulp. As a result much of this document is missing information that will be filled in following their next meeting.

This paper is the problem statement for the Social Device for Non-Technical Users project proposed by Arian Kulp and implemented by Austin Kwong, Haolin Han, and Samuel Bonner. The problem is defined as a need for the optimization of an Android social media application designed for use by non-technical users who may be inexperienced when using mobile applications. The proposed solution of the optimization of the Android application is to streamline navigation on the app, as well as increase the performance speed of application operations. Several metrics for determining success of the project are also outlined, including a measurable increase in the performance of specific application operations (page loading, database access).

\pagebreak

\section*{Problem Definition}

The problem involves the optimization of the user experience, user interface, and performance of an Android social media application that runs solitarily on a device intended for use by non-technical users. This targeted userbase may include those who have little experience operating mobile devices, those who may be unfamiliar navigating the internet and social media applications, the elderly, or children. Operation of the application by users non-technical or otherwise should experience as minimal confusion and frustration from operating this device and app as possible. Modifications and optimizations may be made to the interface and webpage navigation of the device in order to better the usage experience for non-technical and inexperienced users. The social media application is intended to provide a way for non-technical users to connect with friends and family through use of the device and app. (Additional details about the functionality of the app and device will be added after meeting with Arian Kulp. Information may include details of what features the application currently has and what the client may like to see implemented.) 

The client also requests that the application approach native app performance in speed and perhaps stability. The client has indentified they desire maximum performance or optimization of transition times, data loading, caching, and storage. (This section will be expanded upon with more quantifiable data after speaking with client.)

\section*{Proposed Solution}

(Group 29 has yet to have been revealed the details of the Android application or its associated device, so specific performance solutions are not yet available. Proposed solutions regarding user experience optimizations are likewise speculatory.)
Proposed optimizations to the social media application include the replacement of technical terminology and language with clear, everyday language that non-technical users will be able to understand. In addition, all visible text will be displayed in a font size such that the text is readable to all users, including the elderly and those who may be visually impaired. Navigation through the applications features will also be optimized such that non-technical and new users will have minimal confusion about how to access a certain feature. Prompts may be added on certain UI elements, including search bars and on welcome pages in order to guide users to common features, such as friends / family member lists. 

\section*{Metrics}

(Metrics will be discussed with sponsor after meeting.)

\begin{enumerate}
\item New non-technical users are able to create an account and perform basic app operations within a reasonable amount of time, and without expressing frustration.
\item Percentage increase in speed of app feature?
\item Metric 3
\item Metric 4
\end{enumerate}

\end{document}
